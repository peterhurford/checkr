\nonstopmode{}
\documentclass[a4paper]{book}
\usepackage[times,inconsolata,hyper]{Rd}
\usepackage{makeidx}
\usepackage[utf8,latin1]{inputenc}
% \usepackage{graphicx} % @USE GRAPHICX@
\makeindex{}
\begin{document}
\chapter*{}
\begin{center}
{\textbf{\huge Package `checkr'}}
\par\bigskip{\large \today}
\end{center}
\begin{description}
\raggedright{}
\item[Title]\AsIs{Automatically Test R Functions}
\item[Version]\AsIs{0.0.2.9002}
\item[Author]\AsIs{Peter Hurford }\email{peter@peterhurford.com}\AsIs{}
\item[Maintainer]\AsIs{Peter Hurford }\email{peter@peterhurford.com}\AsIs{}
\item[Description]\AsIs{Be able to specify preconditions and postconditions for R functions
and have them automatically checked at runtime. Then, randomly generate R
objects and pass them into R functions, verifying that certain specified
conditions hold.}
\item[Depends]\AsIs{R (>= 3.1.0)}
\item[License]\AsIs{MIT}
\item[LazyData]\AsIs{true}
\item[Suggests]\AsIs{testthat, microbenchmark}
\item[RoxygenNote]\AsIs{5.0.1.9000}
\end{description}
\Rdcontents{\R{} topics documented:}
\inputencoding{utf8}
\HeaderA{ensure}{Ensure checks that certain preconditions and postconditions of a function are true.}{ensure}
%
\begin{Description}\relax
Ensure checks that certain preconditions and postconditions of a function are true.
\end{Description}
%
\begin{Usage}
\begin{verbatim}
ensure(fn, preconditions = list(), postconditions = list())
\end{verbatim}
\end{Usage}
%
\begin{Arguments}
\begin{ldescription}
\item[\code{fn}] function. A function to run with validated pre- and postconditions.

\item[\code{preconditions}] list. A list of preconditions to check.

\item[\code{postconditions}] list. A list of postconditions to check.
\end{ldescription}
\end{Arguments}
%
\begin{Value}
The original function, but also of class validated\_function, with added validations.
\end{Value}
%
\begin{Examples}
\begin{ExampleCode}
  add <- ensure(pre = list(x %is% numeric, y %is% numeric),
    post = list(result %is% numeric),
    function(x, y) { x + y })
\end{ExampleCode}
\end{Examples}
\inputencoding{utf8}
\HeaderA{force\_reload\_test\_objects}{Function to force reload the test object cache, if needed.}{force.Rul.reload.Rul.test.Rul.objects}
%
\begin{Description}\relax
Function to force reload the test object cache, if needed.
\end{Description}
%
\begin{Usage}
\begin{verbatim}
force_reload_test_objects()
\end{verbatim}
\end{Usage}
\inputencoding{utf8}
\HeaderA{function\_name}{Get the name from a passed function, which may be a validated function or just a block.}{function.Rul.name}
%
\begin{Description}\relax
Get the name from a passed function, which may be a validated function or just a block.
\end{Description}
%
\begin{Usage}
\begin{verbatim}
function_name(orig_function_name)
\end{verbatim}
\end{Usage}
%
\begin{Arguments}
\begin{ldescription}
\item[\code{orig\_function\_name.}] A substituted call of the function.
\end{ldescription}
\end{Arguments}
\inputencoding{utf8}
\HeaderA{function\_test\_objects}{Create the necessary testing objects to quickcheck a function.}{function.Rul.test.Rul.objects}
%
\begin{Description}\relax
Create the necessary testing objects to quickcheck a function.
\end{Description}
%
\begin{Usage}
\begin{verbatim}
function_test_objects(fn)
\end{verbatim}
\end{Usage}
%
\begin{Arguments}
\begin{ldescription}
\item[\code{fn}] function. A function to generate test objects for.
\end{ldescription}
\end{Arguments}
\inputencoding{utf8}
\HeaderA{get\_prevalidated\_fn}{Get the pre-validated function that is wrapped in validations.}{get.Rul.prevalidated.Rul.fn}
%
\begin{Description}\relax
Get the pre-validated function that is wrapped in validations.
\end{Description}
%
\begin{Usage}
\begin{verbatim}
get_prevalidated_fn(...)
\end{verbatim}
\end{Usage}
%
\begin{Arguments}
\begin{ldescription}
\item[\code{fn}] validated\_function. The function to get the pre-validated function for.
\end{ldescription}
\end{Arguments}
%
\begin{Value}
a call containing the postconditions.
\end{Value}
\inputencoding{utf8}
\HeaderA{installed\_dataframes}{Get all the user-installed dataframes through data()}{installed.Rul.dataframes}
%
\begin{Description}\relax
Get all the user-installed dataframes through data()
\end{Description}
%
\begin{Usage}
\begin{verbatim}
installed_dataframes()
\end{verbatim}
\end{Usage}
\inputencoding{utf8}
\HeaderA{is.empty}{Tests whether an object is empty.}{is.empty}
%
\begin{Description}\relax
Empty items are NULL, NA, or nothing (length 0).
\end{Description}
%
\begin{Usage}
\begin{verbatim}
is.empty(obj)
\end{verbatim}
\end{Usage}
%
\begin{Arguments}
\begin{ldescription}
\item[\code{obj.}] The object to test.
\end{ldescription}
\end{Arguments}
%
\begin{Value}
a boolean whether or not the object is empty.
\end{Value}
%
\begin{Examples}
\begin{ExampleCode}
  is.empty(NULL)
  is.empty(NA)
  is.empty(list(NULL, NA))
  is.empty(list())
  is.empty(c())
  is.empty(data.frame())
  is.empty("")
  is.empty(data.frame())
\end{ExampleCode}
\end{Examples}
\inputencoding{utf8}
\HeaderA{is.simple\_string}{Tests whether a string is simple.}{is.simple.Rul.string}
%
\begin{Description}\relax
A simple string is an R object that is a length-1 vector of non-empty characters.
\end{Description}
%
\begin{Usage}
\begin{verbatim}
is.simple_string(string)
\end{verbatim}
\end{Usage}
%
\begin{Arguments}
\begin{ldescription}
\item[\code{string}] character.
\end{ldescription}
\end{Arguments}
%
\begin{Value}
a boolean whether or not string is simple string.
\end{Value}
%
\begin{Examples}
\begin{ExampleCode}
  is.simple_string("pizza")              # true
  is.simple_string(c("pizza", "apple"))  # false
  is.simple_string(iris)                 # false
  is.simple_string(NA)                   # false
\end{ExampleCode}
\end{Examples}
\inputencoding{utf8}
\HeaderA{list\_classes}{Get all the classes within a list.}{list.Rul.classes}
%
\begin{Description}\relax
Get all the classes within a list.
\end{Description}
%
\begin{Usage}
\begin{verbatim}
list_classes(object)
\end{verbatim}
\end{Usage}
\inputencoding{utf8}
\HeaderA{postconditions}{Get the stated postconditions of a validated function.}{postconditions}
%
\begin{Description}\relax
Get the stated postconditions of a validated function.
\end{Description}
%
\begin{Usage}
\begin{verbatim}
postconditions(fn)
\end{verbatim}
\end{Usage}
%
\begin{Arguments}
\begin{ldescription}
\item[\code{fn}] validated\_function. The function to get the postconditions for.
\end{ldescription}
\end{Arguments}
%
\begin{Value}
a call containing the postconditions.
\end{Value}
\inputencoding{utf8}
\HeaderA{preconditions}{Get the stated preconditions of a validated function.}{preconditions}
%
\begin{Description}\relax
Get the stated preconditions of a validated function.
\end{Description}
%
\begin{Usage}
\begin{verbatim}
preconditions(fn)
\end{verbatim}
\end{Usage}
%
\begin{Arguments}
\begin{ldescription}
\item[\code{fn}] validated\_function. The function to get the preconditions for.
\end{ldescription}
\end{Arguments}
%
\begin{Value}
a call containing the preconditions.
\end{Value}
\inputencoding{utf8}
\HeaderA{present}{Tests whether an argument to a function is present.}{present}
%
\begin{Description}\relax
This function is the opposite of missing.
\end{Description}
%
\begin{Usage}
\begin{verbatim}
present(...)
\end{verbatim}
\end{Usage}
\inputencoding{utf8}
\HeaderA{print.validated\_function}{Print validated functions more clearly.}{print.validated.Rul.function}
%
\begin{Description}\relax
Print validated functions more clearly.
\end{Description}
%
\begin{Usage}
\begin{verbatim}
## S3 method for class 'validated_function'
print(x, ...)
\end{verbatim}
\end{Usage}
%
\begin{Arguments}
\begin{ldescription}
\item[\code{x}] function. The function to print.

\item[\code{...}] Additional arguments to pass to print.
\end{ldescription}
\end{Arguments}
\inputencoding{utf8}
\HeaderA{print\_args}{Print function arguments}{print.Rul.args}
%
\begin{Description}\relax
Print function arguments
\end{Description}
%
\begin{Usage}
\begin{verbatim}
print_args(x)
\end{verbatim}
\end{Usage}
%
\begin{Examples}
\begin{ExampleCode}
l <- list(x = seq(3), y = seq(4))
print_args(l)
[1] "x = 1:3, y = 1:4"
\end{ExampleCode}
\end{Examples}
\inputencoding{utf8}
\HeaderA{quickcheck}{Quickcheck a function.}{quickcheck}
%
\begin{Description}\relax
Tests a function with many automatically generated inputs, checking that stated
postconditions hold.
\end{Description}
%
\begin{Usage}
\begin{verbatim}
quickcheck(fn, postconditions = NULL, verbose = TRUE, testthat = TRUE)
\end{verbatim}
\end{Usage}
%
\begin{Arguments}
\begin{ldescription}
\item[\code{fn}] function. A function to randomly check postconditions for.

\item[\code{verbose}] logical. Whether or not to announce the success.

\item[\code{testthat}] logical. Whether or not to run testthat.

\item[\code{postconditions.}] Optional postconditions to quickcheck for.
\end{ldescription}
\end{Arguments}
%
\begin{Details}\relax
If given a function of class \code{validated\_function}, the pre- and post-conditions can
be automatically inferred by the definition of the function. The test objects used to
test the function will be screened ahead of time to ensure they meet the preconditions.
\end{Details}
%
\begin{Value}
either TRUE if the function passed the quickcheck or FALSE if it didn't.
\end{Value}
\inputencoding{utf8}
\HeaderA{random\_matrix}{Generate a random matrix.}{random.Rul.matrix}
%
\begin{Description}\relax
A random matrix needs three random things...
A random width, a random height, and a random data
data should be a random assortment of integers, doubles, logicals, or characters, with
all of them being the same class.
Because there are so many possible matricies, it seems easier to generate them on
demand rather than preallocate all possible matricies into OBJECTS.
We will then populate some random matricies onto OBJECTS for later use.
\end{Description}
%
\begin{Usage}
\begin{verbatim}
random_matrix()
\end{verbatim}
\end{Usage}
\inputencoding{utf8}
\HeaderA{random\_objs}{Generate a vector or list of random objects from a particular set of possible choices.}{random.Rul.objs}
%
\begin{Description}\relax
Generate a vector or list of random objects from a particular set of possible choices.
\end{Description}
%
\begin{Usage}
\begin{verbatim}
random_objs(objects, amount)
\end{verbatim}
\end{Usage}
\inputencoding{utf8}
\HeaderA{random\_simple\_strings}{Generate a random simple string (i.e., a length-1 non-empty vector of characters).}{random.Rul.simple.Rul.strings}
%
\begin{Description}\relax
Generate a random simple string (i.e., a length-1 non-empty vector of characters).
\end{Description}
%
\begin{Usage}
\begin{verbatim}
random_simple_strings(amount, chars = TRUE, utf8 = FALSE)
\end{verbatim}
\end{Usage}
\inputencoding{utf8}
\HeaderA{test\_objects\_}{Generates random R objects to be put into functions for testing purposes.}{test.Rul.objects.Rul.}
%
\begin{Description}\relax
Generates random R objects to be put into functions for testing purposes.
\end{Description}
%
\begin{Usage}
\begin{verbatim}
test_objects_()
\end{verbatim}
\end{Usage}
\inputencoding{utf8}
\HeaderA{validate}{Validate checks that certain facts are true.}{validate}
%
\begin{Description}\relax
Validate checks that certain facts are true.
\end{Description}
%
\begin{Usage}
\begin{verbatim}
validate(...)
\end{verbatim}
\end{Usage}
%
\begin{Arguments}
\begin{ldescription}
\item[\code{...}] list. A list of conditions to check.
\end{ldescription}
\end{Arguments}
%
\begin{Value}
Either TRUE or stops with a list of errors.
\end{Value}
%
\begin{Examples}
\begin{ExampleCode}
  validate(1 == 1, "a" %is% character, length(c(1, 2, 3)) == 3)
\end{ExampleCode}
\end{Examples}
\inputencoding{utf8}
\HeaderA{validate\_}{Validate without NSE.}{validate.Rul.}
%
\begin{Description}\relax
Validate without NSE.
\end{Description}
%
\begin{Usage}
\begin{verbatim}
validate_(conditions, env = parent.frame(2))
\end{verbatim}
\end{Usage}
%
\begin{Arguments}
\begin{ldescription}
\item[\code{conditions}] list. A list of conditions to check.

\item[\code{env}] environment. An optional environment to evaluate within. Defaults to
\code{parent.frame(2)}, which contains the variables in the scope immediately beyond
the validate (though not the validate\_) function.
\end{ldescription}
\end{Arguments}
\inputencoding{utf8}
\HeaderA{\Rpercent{}contains\Rpercent{}}{Test if a list contains some elements of the desired class.}{.Rpcent.contains.Rpcent.}
\aliasA{\%contains\_only\%}{\Rpercent{}contains\Rpercent{}}{.Rpcent.contains.Rul.only.Rpcent.}
%
\begin{Description}\relax
Test if a list contains some elements of the desired class.

Test if a list contains only elements of the desired class.
\end{Description}
%
\begin{Usage}
\begin{verbatim}
match_list %contains% expected_class

match_list %contains_only% expected_class
\end{verbatim}
\end{Usage}
%
\begin{Arguments}
\begin{ldescription}
\item[\code{match\_list.}] The list to test for class of the elements.

\item[\code{expected\_class.}] The name of the expected class to test.
\end{ldescription}
\end{Arguments}
%
\begin{Value}
Boolean whether or not the match\_list has at least one element of expected\_class.

Boolean whether or not the match\_list has all elements of expected\_class.
\end{Value}
%
\begin{Examples}
\begin{ExampleCode}
  list(1, 2, 3) %contains% numeric
  list(1, 2, "a") %contains% numeric
  list(1, 2, 3) %contains_only% numeric
  list(1, 2, "a") %contains_only% numeric
\end{ExampleCode}
\end{Examples}
\inputencoding{utf8}
\HeaderA{\Rpercent{}does\_not\_contain\Rpercent{}}{Test if a list does not contain some elements of the desired class.}{.Rpcent.does.Rul.not.Rul.contain.Rpcent.}
%
\begin{Description}\relax
Test if a list does not contain some elements of the desired class.
\end{Description}
%
\begin{Usage}
\begin{verbatim}
... %does_not_contain% NA
\end{verbatim}
\end{Usage}
%
\begin{Arguments}
\begin{ldescription}
\item[\code{match\_list.}] The list to test for class of the elements.

\item[\code{expected\_class.}] The name of the expected class to test.
\end{ldescription}
\end{Arguments}
%
\begin{Value}
Boolean whether or not the match\_list has no elements of the expected\_class.
\end{Value}
%
\begin{Examples}
\begin{ExampleCode}
  list(1, 2, 3) %does_not_contain% character
\end{ExampleCode}
\end{Examples}
\inputencoding{utf8}
\HeaderA{\Rpercent{}is\Rpercent{}}{Test for class membership}{.Rpcent.is.Rpcent.}
\aliasA{\%isnot\%}{\Rpercent{}is\Rpercent{}}{.Rpcent.isnot.Rpcent.}
%
\begin{Description}\relax
Test for class membership

Test whether a match object is not a member of a particular class.
\end{Description}
%
\begin{Usage}
\begin{verbatim}
match_object %is% expected_class

... %isnot% NA
\end{verbatim}
\end{Usage}
%
\begin{Arguments}
\begin{ldescription}
\item[\code{match\_object.}] The object to test for class.

\item[\code{expected\_class.}] The name of the expected class.
\end{ldescription}
\end{Arguments}
%
\begin{Value}
Boolean whether or not the match\_object is the expected\_class.
\end{Value}
%
\begin{Examples}
\begin{ExampleCode}
  1 %is% numeric
  1.0 %is% double
  1L %is% integer
  iris %is% dataframe
  c("a", "b", "c") %is% vector
  "pizza" %is% simple_string
  list(a = "pizza", b = "pie") %is% c("character", "list")
\end{ExampleCode}
\end{Examples}
\printindex{}
\end{document}
